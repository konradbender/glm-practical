\documentclass[a4paper,11pt]{article}

\usepackage{geometry} % for making easy changes to page layout
\geometry{body={15cm,24cm}} % change height and width of main text

\usepackage{parskip} % for blank lines between paragraphs 

\usepackage{amsmath,amssymb} % for serious mathematics, such as \begin{align*} etc

\usepackage{graphicx} % for \includegraphics{} etc

\title{Example document}
\author{Neil Laws}
\date{31 September 2022}

\begin{document}

\maketitle

\section{Introduction}

This is an example Latex document.

\subsection{Maths and equations}

Mathematical symbols are possible, such as $x$, $y$ and $\alpha$.

Including an equation inline is possible, such as $\widehat{\beta} = (X^T X)^{-1}X^T y$.

Also displayed equations:
\begin{equation}
	y = X\beta + \epsilon.
\end{equation}

And aligned displays:
\begin{align*}
	p &= P(|Z| \geq z_0)\\
	  &= 2(1 - \Phi(z_0)).
\end{align*}

\subsection{Figures}

It is also possible to include figures.



The R code that generated \texttt{cars.pdf} was:
\begin{verbatim}
pdf("cars.pdf", width = 5, height = 5)
plot(dist ~ speed, data = cars,
     xlab = "Speed (mph)", ylab = "Stopping Distance (ft)")
dev.off()
\end{verbatim}

In this example the files \texttt{example.tex} and \texttt{cars.pdf} need to be in the same folder -- so that when Latex is run on the file \texttt{example.tex}, the included file \texttt{cars.pdf} is found.

Don't specify a different aspect ratio in Latex to that used when generating the figure. Here's an example of what can happen:



\section{Another section}

If you specify just one of \texttt{height} and \texttt{width}, the aspect ratio will not change:

\includegraphics{../plots/plot.pdf}





\end{document}
